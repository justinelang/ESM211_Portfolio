% Options for packages loaded elsewhere
\PassOptionsToPackage{unicode}{hyperref}
\PassOptionsToPackage{hyphens}{url}
%
\documentclass[
  ignorenonframetext,
]{beamer}
\usepackage{pgfpages}
\setbeamertemplate{caption}[numbered]
\setbeamertemplate{caption label separator}{: }
\setbeamercolor{caption name}{fg=normal text.fg}
\beamertemplatenavigationsymbolsempty
% Prevent slide breaks in the middle of a paragraph
\widowpenalties 1 10000
\raggedbottom
\setbeamertemplate{part page}{
  \centering
  \begin{beamercolorbox}[sep=16pt,center]{part title}
    \usebeamerfont{part title}\insertpart\par
  \end{beamercolorbox}
}
\setbeamertemplate{section page}{
  \centering
  \begin{beamercolorbox}[sep=12pt,center]{part title}
    \usebeamerfont{section title}\insertsection\par
  \end{beamercolorbox}
}
\setbeamertemplate{subsection page}{
  \centering
  \begin{beamercolorbox}[sep=8pt,center]{part title}
    \usebeamerfont{subsection title}\insertsubsection\par
  \end{beamercolorbox}
}
\AtBeginPart{
  \frame{\partpage}
}
\AtBeginSection{
  \ifbibliography
  \else
    \frame{\sectionpage}
  \fi
}
\AtBeginSubsection{
  \frame{\subsectionpage}
}
\usepackage{amsmath,amssymb}
\usepackage{lmodern}
\usepackage{iftex}
\ifPDFTeX
  \usepackage[T1]{fontenc}
  \usepackage[utf8]{inputenc}
  \usepackage{textcomp} % provide euro and other symbols
\else % if luatex or xetex
  \usepackage{unicode-math}
  \defaultfontfeatures{Scale=MatchLowercase}
  \defaultfontfeatures[\rmfamily]{Ligatures=TeX,Scale=1}
\fi
\usetheme[]{Singapore}
% Use upquote if available, for straight quotes in verbatim environments
\IfFileExists{upquote.sty}{\usepackage{upquote}}{}
\IfFileExists{microtype.sty}{% use microtype if available
  \usepackage[]{microtype}
  \UseMicrotypeSet[protrusion]{basicmath} % disable protrusion for tt fonts
}{}
\makeatletter
\@ifundefined{KOMAClassName}{% if non-KOMA class
  \IfFileExists{parskip.sty}{%
    \usepackage{parskip}
  }{% else
    \setlength{\parindent}{0pt}
    \setlength{\parskip}{6pt plus 2pt minus 1pt}}
}{% if KOMA class
  \KOMAoptions{parskip=half}}
\makeatother
\usepackage{xcolor}
\newif\ifbibliography
\usepackage{color}
\usepackage{fancyvrb}
\newcommand{\VerbBar}{|}
\newcommand{\VERB}{\Verb[commandchars=\\\{\}]}
\DefineVerbatimEnvironment{Highlighting}{Verbatim}{commandchars=\\\{\}}
% Add ',fontsize=\small' for more characters per line
\usepackage{framed}
\definecolor{shadecolor}{RGB}{248,248,248}
\newenvironment{Shaded}{\begin{snugshade}}{\end{snugshade}}
\newcommand{\AlertTok}[1]{\textcolor[rgb]{0.94,0.16,0.16}{#1}}
\newcommand{\AnnotationTok}[1]{\textcolor[rgb]{0.56,0.35,0.01}{\textbf{\textit{#1}}}}
\newcommand{\AttributeTok}[1]{\textcolor[rgb]{0.77,0.63,0.00}{#1}}
\newcommand{\BaseNTok}[1]{\textcolor[rgb]{0.00,0.00,0.81}{#1}}
\newcommand{\BuiltInTok}[1]{#1}
\newcommand{\CharTok}[1]{\textcolor[rgb]{0.31,0.60,0.02}{#1}}
\newcommand{\CommentTok}[1]{\textcolor[rgb]{0.56,0.35,0.01}{\textit{#1}}}
\newcommand{\CommentVarTok}[1]{\textcolor[rgb]{0.56,0.35,0.01}{\textbf{\textit{#1}}}}
\newcommand{\ConstantTok}[1]{\textcolor[rgb]{0.00,0.00,0.00}{#1}}
\newcommand{\ControlFlowTok}[1]{\textcolor[rgb]{0.13,0.29,0.53}{\textbf{#1}}}
\newcommand{\DataTypeTok}[1]{\textcolor[rgb]{0.13,0.29,0.53}{#1}}
\newcommand{\DecValTok}[1]{\textcolor[rgb]{0.00,0.00,0.81}{#1}}
\newcommand{\DocumentationTok}[1]{\textcolor[rgb]{0.56,0.35,0.01}{\textbf{\textit{#1}}}}
\newcommand{\ErrorTok}[1]{\textcolor[rgb]{0.64,0.00,0.00}{\textbf{#1}}}
\newcommand{\ExtensionTok}[1]{#1}
\newcommand{\FloatTok}[1]{\textcolor[rgb]{0.00,0.00,0.81}{#1}}
\newcommand{\FunctionTok}[1]{\textcolor[rgb]{0.00,0.00,0.00}{#1}}
\newcommand{\ImportTok}[1]{#1}
\newcommand{\InformationTok}[1]{\textcolor[rgb]{0.56,0.35,0.01}{\textbf{\textit{#1}}}}
\newcommand{\KeywordTok}[1]{\textcolor[rgb]{0.13,0.29,0.53}{\textbf{#1}}}
\newcommand{\NormalTok}[1]{#1}
\newcommand{\OperatorTok}[1]{\textcolor[rgb]{0.81,0.36,0.00}{\textbf{#1}}}
\newcommand{\OtherTok}[1]{\textcolor[rgb]{0.56,0.35,0.01}{#1}}
\newcommand{\PreprocessorTok}[1]{\textcolor[rgb]{0.56,0.35,0.01}{\textit{#1}}}
\newcommand{\RegionMarkerTok}[1]{#1}
\newcommand{\SpecialCharTok}[1]{\textcolor[rgb]{0.00,0.00,0.00}{#1}}
\newcommand{\SpecialStringTok}[1]{\textcolor[rgb]{0.31,0.60,0.02}{#1}}
\newcommand{\StringTok}[1]{\textcolor[rgb]{0.31,0.60,0.02}{#1}}
\newcommand{\VariableTok}[1]{\textcolor[rgb]{0.00,0.00,0.00}{#1}}
\newcommand{\VerbatimStringTok}[1]{\textcolor[rgb]{0.31,0.60,0.02}{#1}}
\newcommand{\WarningTok}[1]{\textcolor[rgb]{0.56,0.35,0.01}{\textbf{\textit{#1}}}}
\usepackage{graphicx}
\makeatletter
\def\maxwidth{\ifdim\Gin@nat@width>\linewidth\linewidth\else\Gin@nat@width\fi}
\def\maxheight{\ifdim\Gin@nat@height>\textheight\textheight\else\Gin@nat@height\fi}
\makeatother
% Scale images if necessary, so that they will not overflow the page
% margins by default, and it is still possible to overwrite the defaults
% using explicit options in \includegraphics[width, height, ...]{}
\setkeys{Gin}{width=\maxwidth,height=\maxheight,keepaspectratio}
% Set default figure placement to htbp
\makeatletter
\def\fps@figure{htbp}
\makeatother
\setlength{\emergencystretch}{3em} % prevent overfull lines
\providecommand{\tightlist}{%
  \setlength{\itemsep}{0pt}\setlength{\parskip}{0pt}}
\setcounter{secnumdepth}{-\maxdimen} % remove section numbering
\ifLuaTeX
  \usepackage{selnolig}  % disable illegal ligatures
\fi
\IfFileExists{bookmark.sty}{\usepackage{bookmark}}{\usepackage{hyperref}}
\IfFileExists{xurl.sty}{\usepackage{xurl}}{} % add URL line breaks if available
\urlstyle{same} % disable monospaced font for URLs
\hypersetup{
  pdftitle={growthrates R package},
  pdfauthor={Linus Blomqvist},
  hidelinks,
  pdfcreator={LaTeX via pandoc}}

\title{growthrates R package}
\subtitle{ESM 211 Winter 2024}
\author{Linus Blomqvist}
\date{}

\begin{document}
\frame{\titlepage}

\begin{frame}[fragile]{What does the package do?}
\protect\hypertarget{what-does-the-package-do}{}
\begin{itemize}
\tightlist
\item
  \texttt{growthrates} offers various ways of assessing growth rates
  (and other growth parameters) in populations
\item
  For example, finding the \(r\) in an exponential growth function, or
  \(r_{max}\) and \(K\) in a logistic growth function
\item
  Can also estimate dose response curves
\item
  It doesn't have its own model-fitting routines, but rather serves as a
  convenient wrapper for other packages
\end{itemize}
\end{frame}

\begin{frame}{Application}
\protect\hypertarget{application}{}
\begin{itemize}
\tightlist
\item
  The models can be applied to any type of population, from microbes to
  bison
\item
  For the simplest applications, all you need is a time series of
  population size
\item
  Can also handle multiple time series from a single experiment (see
  example)
\end{itemize}
\end{frame}

\begin{frame}{Example}
\protect\hypertarget{example}{}
\begin{itemize}
\tightlist
\item
  I will use a dataset on bacterial growth that comes with the package
\item
  Three strains of bacteria (D = Donor, R = Recipient, T =
  transconjugant)
\item
  Different concentrations of the antibiotic tetracycline
\item
  Read off at 31 different times (0:30)
\item
  Two replicates
\end{itemize}
\end{frame}

\begin{frame}[fragile]{Load, inspect, subsample data}
\protect\hypertarget{load-inspect-subsample-data}{}
\begin{Shaded}
\begin{Highlighting}[]
\CommentTok{\# Load dataset and show summary}
\FunctionTok{data}\NormalTok{(bactgrowth)}
\FunctionTok{summary}\NormalTok{(bactgrowth)}

\CommentTok{\# Create subsample to fit graphs on one page}
\NormalTok{bactgrowth\_small }\OtherTok{\textless{}{-}}\NormalTok{ bactgrowth }\SpecialCharTok{\%\textgreater{}\%}
  \FunctionTok{filter}\NormalTok{(conc }\SpecialCharTok{\%in\%} \FunctionTok{sample}\NormalTok{(conc, }\DecValTok{4}\NormalTok{))}
\end{Highlighting}
\end{Shaded}
\end{frame}

\begin{frame}{Plot data}
\protect\hypertarget{plot-data}{}
\includegraphics{growthrates_prez_files/figure-beamer/unnamed-chunk-3-1.pdf}
\end{frame}

\begin{frame}[fragile]{Simplest case: single dataset (time series)}
\protect\hypertarget{simplest-case-single-dataset-time-series}{}
\begin{itemize}
\tightlist
\item
  Three functions: \texttt{fit\_easylinear}, \texttt{fit\_growthmodels},
  and \texttt{fit\_splines}
\item
  For single dataset (just one growth curve), need to subset data
\end{itemize}

\begin{Shaded}
\begin{Highlighting}[]
\CommentTok{\# Subset bactgrowth data}
\NormalTok{splitted.data }\OtherTok{\textless{}{-}} \FunctionTok{multisplit}\NormalTok{(bactgrowth, }
                            \FunctionTok{c}\NormalTok{(}\StringTok{"strain"}\NormalTok{, }\StringTok{"conc"}\NormalTok{, }\StringTok{"replicate"}\NormalTok{))}
\NormalTok{dat }\OtherTok{\textless{}{-}}\NormalTok{ splitted.data[[}\DecValTok{1}\NormalTok{]]}
\end{Highlighting}
\end{Shaded}

This includes only strain D, replicate 1, and concentration 0
\end{frame}

\begin{frame}[fragile]{Inspect dataset for single growth curve}
\protect\hypertarget{inspect-dataset-for-single-growth-curve}{}
\begin{Shaded}
\begin{Highlighting}[]
\FunctionTok{head}\NormalTok{(dat }\SpecialCharTok{\%\textgreater{}\%}
       \FunctionTok{select}\NormalTok{(time, value))}
\end{Highlighting}
\end{Shaded}

This is all the data you need to find \(r\), \(K\), and other
parameters. Make sure time starts at 0.
\end{frame}

\begin{frame}{Plot single growth curve}
\protect\hypertarget{plot-single-growth-curve}{}
\includegraphics{growthrates_prez_files/figure-beamer/unnamed-chunk-6-1.pdf}
\end{frame}

\begin{frame}{Finding \(r\)}
\protect\hypertarget{finding-r}{}
Exponential growth: \[N_t = N_0 e^{rt}\] Derivative of exponential
growth: \[ln(N_t) = ln(N_0) + rt\] Can be evaluated as a linear
function:

\[y = b_0 + b_1 t\]

where \(b_0 = ln(N_0)\) and \(b_1 = r\).
\end{frame}

\begin{frame}[fragile]{Using \texttt{growthrates} to find \(r\)}
\protect\hypertarget{using-growthrates-to-find-r}{}
The function \texttt{fit\_easylinear} finds the maximum growth rate on
the exponential segment of the growth curve.

\texttt{fit\_easylinear} takes four arguments:

\texttt{fit\_easylinear(time,\ y,\ h\ =\ 5,\ quota\ =\ 0.95)}

\begin{itemize}
\tightlist
\item
  \texttt{h} = number of data points (see next slide)
\item
  \texttt{quota} = how much of adjacent data to include
\end{itemize}
\end{frame}

\begin{frame}{How \texttt{fit\_easylinear}'s algorithm works}
\protect\hypertarget{how-fit_easylinears-algorithm-works}{}
\begin{enumerate}
\tightlist
\item
  Fit linear regressions to all subsets of h consecutive data points. If
  for example h = 5, fit a linear regression to points 1 . . . 5, 2 . .
  . 6, 3. . . 7 and so on. The method seeks the highest rate of
  exponential growth, so the dependent variable is log-transformed.
\item
  Find the subset with the highest slope bmax and include also the data
  points of adjacent subsets that have a slope of at least
  \(quota \cdot bmax\), e.g.~all data sets that have at least 95\% of
  the maximum slope.
\item
  Fit a new linear model to the extended data window identified in step
  2.
\end{enumerate}
\end{frame}

\begin{frame}[fragile]{Applying \texttt{fit\_easylinear}}
\protect\hypertarget{applying-fit_easylinear}{}
\begin{Shaded}
\begin{Highlighting}[]
\NormalTok{fit }\OtherTok{\textless{}{-}} \FunctionTok{fit\_easylinear}\NormalTok{(dat}\SpecialCharTok{$}\NormalTok{time, dat}\SpecialCharTok{$}\NormalTok{value)}

\CommentTok{\# Keeping defaults for h and quota}

\FunctionTok{coef}\NormalTok{(fit)}
\end{Highlighting}
\end{Shaded}

\(mumax\) is \(r\).
\end{frame}

\begin{frame}{Plotting the results}
\protect\hypertarget{plotting-the-results}{}
\includegraphics{growthrates_prez_files/figure-beamer/unnamed-chunk-8-1.pdf}
\end{frame}

\begin{frame}[fragile]{Using \texttt{fit\_splines}}
\protect\hypertarget{using-fit_splines}{}
Splines fit models piecewise to a dataset, with a polynomial for each
segment, and some fancy math to make it nice and smooth.

\texttt{growthrates} has the function \texttt{fit\_splines} that can
find \(r\) using this method

\begin{Shaded}
\begin{Highlighting}[]
\NormalTok{res }\OtherTok{\textless{}{-}} \FunctionTok{fit\_spline}\NormalTok{(dat}\SpecialCharTok{$}\NormalTok{time, dat}\SpecialCharTok{$}\NormalTok{value)}

\FunctionTok{coef}\NormalTok{(res)}
\end{Highlighting}
\end{Shaded}
\end{frame}

\begin{frame}{Plotting the result}
\protect\hypertarget{plotting-the-result}{}
\includegraphics{growthrates_prez_files/figure-beamer/unnamed-chunk-10-1.pdf}
\end{frame}

\begin{frame}[fragile]{Fitting parametric nonlinear models}
\protect\hypertarget{fitting-parametric-nonlinear-models}{}
\begin{itemize}
\tightlist
\item
  \texttt{growthrates} can also be used to fit parametric nonlinear
  models with the \texttt{fit\_growthmodel} function
\end{itemize}

fit\_growthmodel(\\
FUN,\\
p,\\
time,\\
y,\\
lower = -Inf,\\
upper = Inf,\\
which = names(p),\\
method = ``Marq'',\\
transform = c(``none'', ``log''), control = NULL,\\
\ldots)
\end{frame}

\begin{frame}[fragile]{Applying \texttt{fit\_growthmodel}}
\protect\hypertarget{applying-fit_growthmodel}{}
\texttt{FUN} defines the growth model; we will use
\texttt{FUN\ =\ grow\_logistic}

\[\frac{dN}{dT} = r_{max} \frac{(K-N)}{K} N\]

\texttt{p} sets start values for the parameters \texttt{y\_0},
\texttt{mumax}, and \texttt{K}, where \texttt{mumax} is equivalent to
\(r_{max}\). Make your best guess based on an inspection of the data.

The other variables have defaults and we won't worry about them now.
\end{frame}

\begin{frame}[fragile]{Applying \texttt{fit\_growthmodel}}
\protect\hypertarget{applying-fit_growthmodel-1}{}
\begin{Shaded}
\begin{Highlighting}[]
\NormalTok{p }\OtherTok{\textless{}{-}} \FunctionTok{c}\NormalTok{(}\AttributeTok{y0 =} \FloatTok{0.01}\NormalTok{, }\AttributeTok{mumax =} \FloatTok{0.2}\NormalTok{, }\AttributeTok{K =} \FloatTok{0.1}\NormalTok{)}

\NormalTok{fit1 }\OtherTok{\textless{}{-}} \FunctionTok{fit\_growthmodel}\NormalTok{(}\AttributeTok{FUN =}\NormalTok{ grow\_logistic, }
                        \AttributeTok{p =}\NormalTok{ p, }
\NormalTok{                        dat}\SpecialCharTok{$}\NormalTok{time, }
\NormalTok{                        dat}\SpecialCharTok{$}\NormalTok{value)}

\FunctionTok{coef}\NormalTok{(fit1)}
\end{Highlighting}
\end{Shaded}
\end{frame}

\begin{frame}{Plotting results of logistic growth model}
\protect\hypertarget{plotting-results-of-logistic-growth-model}{}
\includegraphics{growthrates_prez_files/figure-beamer/unnamed-chunk-12-1.pdf}
\end{frame}

\begin{frame}{Two-step differential equation model}
\protect\hypertarget{two-step-differential-equation-model}{}
Cells (or individuals) can be either inactive or active. For animals,
inactive could mean an individual is not of reproductive age.

Each type has its own growth equation:

\[\frac{dy_i}{dt} = - k_w \cdot y_i\]
\[\frac{dy_a}{dt} = k_w \cdot y_i + \mu_{max} \cdot y_a \cdot \left( 1 - \frac{y_a+y_i}{K} \right)\]
\end{frame}

\begin{frame}[fragile]{Applying the two-step differential equation
model}
\protect\hypertarget{applying-the-two-step-differential-equation-model}{}
\begin{Shaded}
\begin{Highlighting}[]
\NormalTok{p }\OtherTok{\textless{}{-}} \FunctionTok{c}\NormalTok{(}\AttributeTok{yi =} \FloatTok{0.02}\NormalTok{, }\AttributeTok{ya =} \FloatTok{0.001}\NormalTok{, }\AttributeTok{kw =} \FloatTok{0.1}\NormalTok{, }\AttributeTok{mumax =} \FloatTok{0.2}\NormalTok{, }\AttributeTok{K =} \FloatTok{0.1}\NormalTok{)}

\NormalTok{fit2 }\OtherTok{\textless{}{-}} \FunctionTok{fit\_growthmodel}\NormalTok{(}\AttributeTok{FUN =}\NormalTok{ grow\_twostep, }
                        \AttributeTok{p =}\NormalTok{ p, }
                        \AttributeTok{time =}\NormalTok{ dat}\SpecialCharTok{$}\NormalTok{time, }
                        \AttributeTok{y =}\NormalTok{ dat}\SpecialCharTok{$}\NormalTok{value)}

\FunctionTok{coef}\NormalTok{(fit2)}
\end{Highlighting}
\end{Shaded}
\end{frame}

\begin{frame}{Plotting result of two-step model}
\protect\hypertarget{plotting-result-of-two-step-model}{}
\includegraphics{growthrates_prez_files/figure-beamer/unnamed-chunk-14-1.pdf}
\end{frame}

\begin{frame}[fragile]{Fitting models to multiple datasets}
\protect\hypertarget{fitting-models-to-multiple-datasets}{}
This applies when you have more than one growth curve. In
\texttt{bactgrowth} we have a total of 72 growth curves, a combination
of three strains, two replicates, and 12 different concentrations.
\end{frame}

\begin{frame}{Full \texttt{bactgrowth} dataset}
\protect\hypertarget{full-bactgrowth-dataset}{}
Remember this plot of a subset with only four concentrations:

\includegraphics{growthrates_prez_files/figure-beamer/unnamed-chunk-15-1.pdf}
\end{frame}

\begin{frame}[fragile]{Analysis of multiple datasets}
\protect\hypertarget{analysis-of-multiple-datasets}{}
One question we might ask ourselves is: how does the concentration of an
antibiotic affect growth rates of different strains and replicates? This
is known as dose response curves. The same logic applies to, for
example, environmental toxins.

We start with a spline fit. \texttt{spar} is a parameter that determines
smoothness of the spline fit. The output gives us estimates for \(y_0\)
and \(r_{max}\).

\begin{Shaded}
\begin{Highlighting}[]
\NormalTok{many\_spline\_fits }\OtherTok{\textless{}{-}} \FunctionTok{all\_splines}\NormalTok{(value }\SpecialCharTok{\textasciitilde{}}\NormalTok{ time }\SpecialCharTok{|} 
\NormalTok{                                  strain }\SpecialCharTok{+}\NormalTok{ conc }\SpecialCharTok{+}\NormalTok{ replicate,}
                                \AttributeTok{data =}\NormalTok{ bactgrowth, }\AttributeTok{spar =} \FloatTok{0.5}\NormalTok{)}
\end{Highlighting}
\end{Shaded}
\end{frame}

\begin{frame}[fragile]{Look at coefficients}
\protect\hypertarget{look-at-coefficients}{}
\begin{Shaded}
\begin{Highlighting}[]
\FunctionTok{head}\NormalTok{(}\FunctionTok{coef}\NormalTok{(many\_spline\_fits), }\DecValTok{5}\NormalTok{)}
\end{Highlighting}
\end{Shaded}
\end{frame}

\begin{frame}{Plotting results for multiple splines}
\protect\hypertarget{plotting-results-for-multiple-splines}{}
\includegraphics{growthrates_prez_files/figure-beamer/unnamed-chunk-19-1.pdf}
\end{frame}

\begin{frame}{Logistic growth models for multiple datasets}
\protect\hypertarget{logistic-growth-models-for-multiple-datasets}{}
all\_growthmodels(\\
formula,\\
data,\\
p,\\
lower = -Inf,\\
upper = Inf,\\
which = names(p),\\
FUN = NULL,\\
method = ``Marq'',\\
transform = c(``none'', ``log''),\\
\ldots,\\
subset = NULL,\\
ncores = detectCores(logical = FALSE)\\
)
\end{frame}

\begin{frame}[fragile]{Applying \texttt{all\_growthmodels} with logistic
growth}
\protect\hypertarget{applying-all_growthmodels-with-logistic-growth}{}
\begin{Shaded}
\begin{Highlighting}[]
\NormalTok{p   }\OtherTok{\textless{}{-}} \FunctionTok{c}\NormalTok{(}\AttributeTok{y0 =} \FloatTok{0.03}\NormalTok{, }\AttributeTok{mumax =}\NormalTok{ .}\DecValTok{1}\NormalTok{, }\AttributeTok{K =} \FloatTok{0.1}\NormalTok{)}

\NormalTok{many\_logistic }\OtherTok{\textless{}{-}} \FunctionTok{all\_growthmodels}\NormalTok{(}
\NormalTok{                   value }\SpecialCharTok{\textasciitilde{}} \FunctionTok{grow\_logistic}\NormalTok{(time, parms) }\SpecialCharTok{|} 
\NormalTok{                     strain }\SpecialCharTok{+}\NormalTok{ conc }\SpecialCharTok{+}\NormalTok{ replicate,}
                   \AttributeTok{data =}\NormalTok{ bactgrowth,}
                   \AttributeTok{p =}\NormalTok{ p)}
\end{Highlighting}
\end{Shaded}
\end{frame}

\begin{frame}{Plotting results}
\protect\hypertarget{plotting-results}{}
\includegraphics{growthrates_prez_files/figure-beamer/unnamed-chunk-21-1.pdf}
\end{frame}

\begin{frame}[fragile]{Using results to estimate dose response curves}
\protect\hypertarget{using-results-to-estimate-dose-response-curves}{}
\begin{Shaded}
\begin{Highlighting}[]
\NormalTok{many\_spline\_res   }\OtherTok{\textless{}{-}} \FunctionTok{results}\NormalTok{(many\_spline\_fits)}
\NormalTok{many\_logistic\_res }\OtherTok{\textless{}{-}} \FunctionTok{results}\NormalTok{(many\_logistic)}
\end{Highlighting}
\end{Shaded}
\end{frame}

\begin{frame}{Plotting dose response curves for splines}
\protect\hypertarget{plotting-dose-response-curves-for-splines}{}
\includegraphics{growthrates_prez_files/figure-beamer/unnamed-chunk-23-1.pdf}
\end{frame}

\begin{frame}{Plotting dose response curves for logistic}
\protect\hypertarget{plotting-dose-response-curves-for-logistic}{}
\includegraphics{growthrates_prez_files/figure-beamer/unnamed-chunk-24-1.pdf}
\end{frame}

\begin{frame}{Unique insight}
\protect\hypertarget{unique-insight}{}
You can write your own functions!

Example for a shift model (won't go into details of the model)

\[y(t) = \frac{K \cdot y_0}{y_0 + (K-y_0) \cdot e^{-\mu_{max}t}} + y_{shift}\]
\end{frame}

\begin{frame}[fragile]{Code}
\protect\hypertarget{code}{}
\begin{Shaded}
\begin{Highlighting}[]
\CommentTok{\# Define model}
\NormalTok{grow\_logistic\_yshift }\OtherTok{\textless{}{-}} \ControlFlowTok{function}\NormalTok{(time, parms) \{}
  \FunctionTok{with}\NormalTok{(}\FunctionTok{as.list}\NormalTok{(parms), \{}
\NormalTok{    y }\OtherTok{\textless{}{-}}\NormalTok{ (K }\SpecialCharTok{*}\NormalTok{ y0) }\SpecialCharTok{/}\NormalTok{ (y0 }\SpecialCharTok{+}\NormalTok{ (K }\SpecialCharTok{{-}}\NormalTok{ y0) }\SpecialCharTok{*} \FunctionTok{exp}\NormalTok{(}\SpecialCharTok{{-}}\NormalTok{mumax }\SpecialCharTok{*}\NormalTok{ time)) }\SpecialCharTok{+}\NormalTok{ y\_shift}
    \FunctionTok{as.matrix}\NormalTok{(}\FunctionTok{data.frame}\NormalTok{(}\AttributeTok{time =}\NormalTok{ time, }\AttributeTok{y =}\NormalTok{ y))}
\NormalTok{  \})}
\NormalTok{\}}

\CommentTok{\# Convert to class \textasciigrave{}growthmodel\textasciigrave{}}
\NormalTok{grow\_logistic\_yshift }\OtherTok{\textless{}{-}} \FunctionTok{growthmodel}\NormalTok{(grow\_logistic\_yshift,}
                                    \FunctionTok{c}\NormalTok{(}\StringTok{"y0"}\NormalTok{, }\StringTok{"mumax"}\NormalTok{, }\StringTok{"K"}\NormalTok{, }\StringTok{"y\_shift"}\NormalTok{))}

\CommentTok{\# Now use \textasciigrave{}growthrates\textasciigrave{} function to fit model}
\CommentTok{\# fit \textless{}{-} fit\_growthmodel(grow\_logistic\_yshift,}
                       \CommentTok{\# p = c(y0 = 1, mumax = 0.1, K = 10, K = 10, }
                       \CommentTok{\#       y\_shift = 1),}
                       \CommentTok{\# time = x, y = y)}
\end{Highlighting}
\end{Shaded}
\end{frame}

\begin{frame}{Advantages of \texttt{growthrates} package}
\protect\hypertarget{advantages-of-growthrates-package}{}
\begin{itemize}
\tightlist
\item
  Adapted to problems of biological growth
\item
  A lot simpler to use than the packages that contain the fitting
  routines
\item
  Very flexible -- can fit several different models; each function can
  be adapted
\item
  Seems relatively fast
\item
  Allows for user-defined functions
\item
  Has built-in plotting function but also works well with ggplot
\item
  Well documented
\end{itemize}
\end{frame}

\begin{frame}{Limitations of the package}
\protect\hypertarget{limitations-of-the-package}{}
\begin{itemize}
\tightlist
\item
  Would be even easier if some of the user-defined functions (like the
  shift model) were integrated into the package
\item
  Overall the package does what it sets out to do, and I have no major
  complaints
\end{itemize}
\end{frame}

\begin{frame}{Sources}
\protect\hypertarget{sources}{}
\url{https://tpetzoldt.github.io/growthrates/doc/Introduction.html}
\url{https://cran.r-project.org/web/packages/growthrates/growthrates.pdf}
\url{https://github.com/tpetzoldt/growthrates}

My code:

\url{https://github.com/linusblomqvist/ESM_211/blob/main/growthrates/growthrates_prez.Rmd}
\end{frame}

\begin{frame}[fragile]{Portfolio assignment}
\protect\hypertarget{portfolio-assignment}{}
Find \(r\) and \(K\) for the bison dataset using at least three
different models (e.g.~exponential, logistic, spline).

Hint: try functions \texttt{fit\_easylinear},
\texttt{fit\_growthmodels}, and \texttt{fit\_splines}.

Do the different models agree on the parameter values?

\begin{Shaded}
\begin{Highlighting}[]
\DocumentationTok{\#\#\# Read in bison data}

\NormalTok{bison }\OtherTok{\textless{}{-}} \FunctionTok{read\_csv}\NormalTok{(}\FunctionTok{here}\NormalTok{(}\StringTok{"growthrates"}\NormalTok{, }\StringTok{"bison.csv"}\NormalTok{)) }\SpecialCharTok{\%\textgreater{}\%} 
  \FunctionTok{mutate}\NormalTok{(}\AttributeTok{time =}\NormalTok{ year }\SpecialCharTok{{-}}\NormalTok{ year[}\DecValTok{1}\NormalTok{])}
\end{Highlighting}
\end{Shaded}

\begin{Shaded}
\begin{Highlighting}[]
\NormalTok{bison\_fit }\OtherTok{\textless{}{-}} \FunctionTok{fit\_easylinear}\NormalTok{(bison}\SpecialCharTok{$}\NormalTok{time, bison}\SpecialCharTok{$}\NormalTok{bison)}

\CommentTok{\# Keeping defaults for h and quota}

\FunctionTok{coef}\NormalTok{(bison\_fit)}
\end{Highlighting}
\end{Shaded}

\begin{Shaded}
\begin{Highlighting}[]
\FunctionTok{par}\NormalTok{(}\AttributeTok{mfrow =} \FunctionTok{c}\NormalTok{(}\DecValTok{1}\NormalTok{, }\DecValTok{2}\NormalTok{))}
\FunctionTok{plot}\NormalTok{(bison\_fit, }\AttributeTok{log =} \StringTok{"y"}\NormalTok{, }\AttributeTok{cex.lab =} \DecValTok{2}\NormalTok{, }\AttributeTok{cex.axis =} \DecValTok{2}\NormalTok{)}
\FunctionTok{plot}\NormalTok{(bison\_fit, }\AttributeTok{cex.lab =} \DecValTok{2}\NormalTok{, }\AttributeTok{cex.axis =} \DecValTok{2}\NormalTok{)}
\end{Highlighting}
\end{Shaded}

\includegraphics{growthrates_prez_files/figure-beamer/unnamed-chunk-28-1.pdf}

\begin{Shaded}
\begin{Highlighting}[]
\NormalTok{bison\_res }\OtherTok{\textless{}{-}} \FunctionTok{fit\_spline}\NormalTok{(bison}\SpecialCharTok{$}\NormalTok{time, bison}\SpecialCharTok{$}\NormalTok{bison)}

\FunctionTok{coef}\NormalTok{(bison\_res)}
\end{Highlighting}
\end{Shaded}

\begin{Shaded}
\begin{Highlighting}[]
\FunctionTok{par}\NormalTok{(}\AttributeTok{mfrow =} \FunctionTok{c}\NormalTok{(}\DecValTok{1}\NormalTok{, }\DecValTok{2}\NormalTok{))}
\FunctionTok{plot}\NormalTok{(bison\_res, }\AttributeTok{log =} \StringTok{"y"}\NormalTok{, }\AttributeTok{cex.lab =} \DecValTok{2}\NormalTok{, }\AttributeTok{cex.axis =} \DecValTok{2}\NormalTok{)}
\FunctionTok{plot}\NormalTok{(bison\_res, }\AttributeTok{cex.lab =} \DecValTok{2}\NormalTok{, }\AttributeTok{cex.axis =} \DecValTok{2}\NormalTok{)}
\end{Highlighting}
\end{Shaded}

\includegraphics{growthrates_prez_files/figure-beamer/unnamed-chunk-30-1.pdf}

\begin{Shaded}
\begin{Highlighting}[]
\NormalTok{bison\_p }\OtherTok{\textless{}{-}} \FunctionTok{c}\NormalTok{(}\AttributeTok{y0 =} \DecValTok{484}\NormalTok{, }\AttributeTok{mumax =} \FloatTok{0.2}\NormalTok{, }\AttributeTok{K =} \DecValTok{4114}\NormalTok{)}

\NormalTok{bison\_fit1 }\OtherTok{\textless{}{-}} \FunctionTok{fit\_growthmodel}\NormalTok{(}\AttributeTok{FUN =}\NormalTok{ grow\_logistic, }
                        \AttributeTok{p =}\NormalTok{ bison\_p, }
\NormalTok{                        bison}\SpecialCharTok{$}\NormalTok{time, }
\NormalTok{                        bison}\SpecialCharTok{$}\NormalTok{bison)}

\FunctionTok{coef}\NormalTok{(bison\_fit1)}
\end{Highlighting}
\end{Shaded}
\end{frame}

\end{document}
